\documentclass[11pt]{report}

% Paquetes y configuraciones adicionales
\usepackage{graphicx}
\usepackage[export]{adjustbox}
\usepackage{caption}
\usepackage{float}
\usepackage{titlesec}
\usepackage{geometry}
\usepackage[hidelinks]{hyperref}
\usepackage{titling}
\usepackage{titlesec}
\usepackage{parskip}
\usepackage{wasysym}
\usepackage{tikzsymbols}
\usepackage{fancyvrb}
\usepackage{xurl}
\usepackage{hyperref}
\usepackage[spanish]{babel}
\usepackage{listings}
\usepackage{subcaption}
\usepackage{xcolor}

\newcommand{\subtitle}[1]{
  \posttitle{
    \par\end{center}
    \begin{center}\large#1\end{center}
    \vskip0.5em}
}

% Configura los márgenes
\geometry{
  left=2cm,   % Ajusta este valor al margen izquierdo deseado
  right=2cm,  % Ajusta este valor al margen derecho deseado
  top=3cm,
  bottom=3cm,
}

% Configuración de los títulos de las secciones
\titlespacing{\section}{0pt}{\parskip}{\parskip}
\titlespacing{\subsection}{0pt}{\parskip}{\parskip}
\titlespacing{\subsubsection}{0pt}{\parskip}{\parskip}

% Redefinir el formato de los capítulos y añadir un punto después del número
\makeatletter
\renewcommand{\@makechapterhead}[1]{%
  \vspace*{0\p@} % Ajusta este valor para el espaciado deseado antes del título del capítulo
  {\parindent \z@ \raggedright \normalfont
    \ifnum \c@secnumdepth >\m@ne
        \huge\bfseries \thechapter.\ % Añade un punto después del número
    \fi
    \interlinepenalty\@M
    #1\par\nobreak
    \vspace{10pt} % Ajusta este valor para el espacio deseado después del título del capítulo
  }}
\makeatother

% Configura para que cada \chapter no comience en una pagina nueva
\makeatletter
\renewcommand\chapter{\@startsection{chapter}{0}{\z@}%
    {-3.5ex \@plus -1ex \@minus -.2ex}%
    {2.3ex \@plus.2ex}%
    {\normalfont\Large\bfseries}}
\makeatother

% Configurar los colores para el código
\definecolor{codegreen}{rgb}{0,0.6,0}
\definecolor{codegray}{rgb}{0.5,0.5,0.5}
\definecolor{codepurple}{rgb}{0.58,0,0.82}
\definecolor{backcolour}{rgb}{0.95,0.95,0.92}

% Configurar el estilo para el código
\lstdefinestyle{mystyle}{
  backgroundcolor=\color{backcolour},   
  commentstyle=\color{codegreen},
  keywordstyle=\color{magenta},
  numberstyle=\tiny\color{codegray},
  stringstyle=\color{codepurple},
  basicstyle=\ttfamily\footnotesize,
  breakatwhitespace=false,         
  breaklines=true,                 
  captionpos=b,                    
  keepspaces=true,                 
  numbers=left,                    
  numbersep=5pt,                  
  showspaces=false,                
  showstringspaces=false,
  showtabs=false,                  
  tabsize=2
}

%==============================================================================
% Cosas para la documentación LateX
% % Sangría
% \setlength{\parindent}{1em}Texto

% % Quitar sangría
% \noindent

% % Punto
% \CIRCLE \ \ \textbf{Texto} \emph{algo}
% \begin{itemize}
%   \item \textbf{Negrita:} Texto
%   \item \textbf{Negrita:} Texto
% \end{itemize}

% % Introducir código
% \begin{center}
%   \begin{BVerbatim}
%     ... Código
%   \end{BVerbatim}
% \end{center}

% Poner una imagen
% \begin{figure}[H]
%   \centering
%   \includegraphics[scale=0.55]{img/}
%   \caption{Exportación de la base de datos en formato sql}
%   \label{fig:exportación de la base de datos en formato sql}
% \end{figure}

% Poner dos imágenes
% \begin{figure}[H]
%   \begin{subfigure}{0.5\textwidth}
%     \centering
%     \includegraphics[scale=0.45]{img/}
%     \caption{Texto imagen 1}
%   \end{subfigure}%
%   \begin{subfigure}{0.5\textwidth}
%     \centering
%     \includegraphics[scale=0.45]{img/}
%     \caption{Texto imagen 2}
%   \end{subfigure}
%   \caption{Texto general}
% \end{figure}

% % Poner una tabla
% \begin{table}[H]
%   \centering
%   \begin{tabular}{|c|c|c|c|}
%     \hline
%     \textbf{Campo 1} & \textbf{Campo 2} & \textbf{Campo 3} & \textbf{Campo 4} \\ \hline
%     Texto & Texto & Texto & Texto \\ \hline
%     Texto & Texto & Texto & Texto \\ \hline
%     Texto & Texto & Texto & Texto \\ \hline
%     Texto & Texto & Texto & Texto \\ \hline
%   \end{tabular}
%   \caption{Nombre de la tabla}
%   \label{tab:nombre de la tabla}
% \end{table}

%==============================================================================

\begin{document}

% Portada del informe
\title{Practica 09. Shorewall: Doble firewall con DMZ}
\subtitle{Seguridad de Sistemas Informáticos}
\author{Carlos Pérez Fino y Cheuk Kelly Ng Pante}
\date{\today}

\maketitle

\pagestyle{empty} % Desactiva la numeración de página para el índice

% Índice
\tableofcontents

% Nueva página
\cleardoublepage

\pagestyle{plain} % Vuelve a activar la numeración de página
\setcounter{page}{1} % Reinicia el contador de página a 1

% Capitulo 1
\chapter{Configuración de red con dos firewalls y tres zonas}
Esta práctica se va a realizar una configuracion de un firewall con DMZ utilizando
\emph{Shorewall} y \emph{firewalld}. Se va a implementar un diseño con doble firewall
(interno con \emph{firewalld} y externo con \emph{Shorewall}) con dos interfaces para
gestionar las zonas de Internet, DMZ y LAN. La DMZ se localiza entre los dos firewalls 
configurados.

Se va a partir del siguiente diseño de red con dos firewalls y tres zonas:
\begin{figure}[H]
  \centering
  \includegraphics[scale=0.392]{img/esquema_red.png}
  \caption{Diseño de red con dos firewalls y tres zonas}
  \label{fig:Diseno de red con dos firewalls y tres zonas}
\end{figure}

Esta red tendrá tres zonas: \emph{priv} para la red interna, \emph{fw} para el firewall
y \emph{dmz} para la DMZ, con el siguiente direccionamiento:
\begin{itemize}
  \item \textbf{Internet:} la red especificada por el servidor DHCP externo.
  \item \textbf{Red Interna:} Clase C privada como subred de una clase B privada: 172.16.X.0/24.
  \item \textbf{DMZ:} Clase C privada 192.168.X.0/24.
\end{itemize}
\section{Configuración de la red en el firewall externo}
Para la configuración de la red en el firewall externo, se va a configurar la interfaz que
va conectada a la DMZ, para ello se va a configurar el archivo \emph{/etc/network/interfaces}
con la siguiente configuración:
\begin{verbatim}
auto ens4
iface ens4 inet static
        address 192.168.10.2
        netmask 255.255.255.0
\end{verbatim}

Una vez configurada la interfaz, se va reiniciar el servicio de red con el siguiente comando: \\
\begin{BVerbatim}
sudo systemctl restart networking
\end{BVerbatim}

% Nueva página
\cleardoublepage

\section{Configuración de la red en el firewall interno}
Para la configuración de la red en el firewall interno, se va a configurar dos interfaces, una
que va conectada a la DMZ y otra que va conectada a la red interna. Como esta máquina es un \emph{CentOS 8},
la configuración de la red lo haremos con \emph{nmtui}. Para la instalación de \emph{nmtui}, se va a utilizar
el siguiente comando:
\begin{BVerbatim}
sudo yum install NetworkManager-tui
\end{BVerbatim}

Ya instalado, iniciamos el servicio con el siguiente comando:
\begin{BVerbatim}
sudo systemctl start NetworkManager
\end{BVerbatim}

Una vez instalado \emph{nmtui}, se va a configurar la interfaz que va conectada a la DMZ y a la red interna, queda de la siguiente
manera:
\begin{figure}[H]
  \begin{subfigure}{0.5\textwidth}
    \centering
    \includegraphics[scale=0.42]{img/fw-interno_interface_to_DMZ.png}
  \end{subfigure}%
  \begin{subfigure}{0.5\textwidth}
    \centering
    \includegraphics[scale=0.42]{img/fw-interno_interface_to_LAN.png}
  \end{subfigure}
  \caption{Configuración de las interfaces en el firewall interno}
\end{figure}


% % Nueva página
% \cleardoublepage

\section{Resultado de la configuración de la red en el firewall externo e interno}
Una vez configurada la red en el firewall externo e interno, se va a comprobar que la configuración
se ha realizado correctamente. Para ello, se va a utilizar el comando \emph{ip a} en ambos firewalls,
quedando de la siguiente manera:
% Poner dos imágenes
\begin{figure}[H]
  \begin{subfigure}{0.5\textwidth}
    \centering
    \includegraphics[scale=0.35]{img/ip_a_fw_externo.png}
    \caption{Configuracion de la red en el firewall externo}
  \end{subfigure}%
  \begin{subfigure}{0.5\textwidth}
    \centering
    \includegraphics[scale=0.38]{img/ip_a_fw_interno.png}
    \caption{Configuracion de la red en el firewall interno}
  \end{subfigure}
  \caption{Resultado de la configuración de la red en el firewall externo e interno}
\end{figure}

Una vez hecha la configuración de la red, se va a borrar las interfaces externas por defecto
en el servidor, en el cliente y en el firewall interno.

% Nueva página
\cleardoublepage

% Capitulo 2
\chapter{Habilitar \emph{NAT} utilizando la configuración de \emph{Shorewall}}
Antes de empezar con al configuración del \emph{NAT}, hay que asegurarse que en los dos firewalls esté
habilitado el reenvío de paquetes IP o \emph{IP forwarding}. Para ello, se va a utilizar el siguiente comando:
\begin{BVerbatim}
echo 1 | sudo tee /proc/sys/net/ipv4/ip_forward
\end{BVerbatim}


Para habilitar \emph{NAT}, lo haremos en el firewall externo ya que es el que está conectado 
a Internet. Para hacerlo usaremos \emph{Shorewall}, que describe los requisitos de firewall utilizando entradas en un 
conjunto de archivos de configuración. Shorewall lee esos archivos de configuración y, 
con la ayuda de las utilidades \emph{iptables}, \emph{iptables-restore}, \emph{ip} y \emph{tc} configura 
el \emph{Netfilter} y el tráfico de red relacionado de acuerdo con esos requisitos.

La instalación de este programa se va a utilizar el siguiente comando:
\begin{BVerbatim}
sudo apt install shorewall
\end{BVerbatim}

Para habilitar el \emph{forwarding} lo haremos configurando el
fichero \emph{/etc/shorewall/shorewall.conf} con la siguiente configuración:
\begin{figure}[H]
  \centering
  \includegraphics[scale=0.8]{img/forwarding_shorewall.png}
  \caption{Configuración de \emph{forwarding} en \emph{Shorewall}}
  \label{fig:Configuracion de forwarding en Shorewall}
\end{figure}

Una vez habilitado el \emph{forwarding}, se va a configurar los diferentes archivos de configuracion
de \emph{Shorewall}.
En este caso, al instalar \emph{Shorewall} en el firewall externo y como es una máquina \emph{Debian}, no crea los ficheros
de configuración por defecto, por lo que hay que crearlos. Creamos dentro del directorio \emph{/etc/shorewall/} los siguientes
archivos de configuración:
\begin{itemize}
  \item \textbf{zones:} declara las zonas de red.
  \begin{figure}[H]
    \centering
    \includegraphics[scale=0.7]{img/zones.png}
    \caption{Configuración de \emph{/etc/shorewall/zones}}
    \label{fig:Configuracion de /etc/shorewall/zones}
  \end{figure}

  % Nueva página
  \cleardoublepage

  \item \textbf{interfaces:} define las interfaces de red del firewall.
  \begin{figure}[H]
    \centering
    \includegraphics[scale=0.65]{img/interfaces.png}
    \caption{Configuración de \emph{/etc/shorewall/interfaces}}
    \label{fig:Configuracion de /etc/shorewall/interfaces}
  \end{figure}
  \item \textbf{hosts:} define zonas en terminos de subredes y/o direcciones IP individuales.
  \begin{figure}[H]
    \centering
    \includegraphics[scale=0.7]{img/hosts.png}
    \caption{Configuración de \emph{/etc/shorewall/hosts}}
    \label{fig:Configuracion de /etc/shorewall/hosts}
  \end{figure}
  \item \textbf{snat:} contiene las definiciones de \emph{SNAT}. 
  \begin{figure}[H]
    \centering
    \includegraphics[scale=0.65]{img/snat_config.png}
    \caption{Configuración de \emph{/etc/shorewall/snat}}
    \label{fig:Configuracion de /etc/shorewall/snat}
  \end{figure}
\end{itemize}

% Nueva página
\cleardoublepage

% Capitulo 3
\chapter{Configurar el cliente en la red interna y servidor en la DMZ}
\section{Configuración del cliente en la red interna}
Para configurar el cliente en la red interna, se va a configurar el archivo el archivo \emph{/etc/network/interfaces}
con la siguiente configuración:
\begin{verbatim}
auto ens4
iface ens4 inet static
      address 172.16.10.2
      netmask 255.255.255.0
      gateway 172.16.10.1
\end{verbatim}

\section{Configuración del servidor en la DMZ}
Para configurar el servidor en la DMZ, primero vamos a configurar la interfaz que va conectada a la DMZ, para ello se va a configurar el archivo \emph{/etc/network/interfaces}
con la siguiente configuración:
\begin{verbatim}
auto ens4
iface ens4 inet static
      address 192.168.10.100
      netmask 255.255.255.0
      gateway 192.168.10.2
\end{verbatim}

y luego se va a instalar el servicio web con el siguiente comando:
\begin{BVerbatim}
sudo apt install nginx
\end{BVerbatim}

Ahora, se va a configurar el archivo \emph{/etc/nginx/sites-available/default} y añadimos
el siguiente contenido:
\begin{verbatim}
server {
  listen 192.168.10.100:80;
  server_name 10.6.129.75;
}
\end{verbatim}

Una vez configurado el archivo, se va a reiniciar el servicio \emph{nginx} con el siguiente comando: \\
\begin{BVerbatim}
sudo systemctl restart nginx
\end{BVerbatim}

A continuación, se va a comprobar que el servicio \emph{nginx} está funcionando correctamente, para ello se vamos a utilizar un navegador de texto 
en el firewall externo, aqui una captura de pantalla del resultado:
\begin{figure}[H]
  \centering
  \includegraphics[scale=0.55]{img/nginx_fw_externo.png}
  \caption{\emph{nginx} en el firewall externo}
  \label{fig:nginx en el firewall externo}
\end{figure}


Ya con el servicio \emph{nginx} configurado, se va a instalar el servicio \emph{proftpd} para tener un servidor FTP. Para su instalación
se va a utilizar el siguiente comando:
\begin{BVerbatim}
sudo apt install proftpd
\end{BVerbatim}

Con el servicio \emph{proftpd} instalado, se va a iniciar el servicio:
\begin{BVerbatim}
systemctl start proftpd
\end{BVerbatim}

Ahora se va a probar el funcionamiento del servidor FTP, para ello se va a utilizar el comando \emph{ftp} firewall externo, aqui una
captura de pantalla del resultado:
\begin{figure}[H]
  \centering
  \includegraphics[scale=0.55]{img/ftp.png}
  \caption{Resultado de la prueba del servidor FTP en el firewall externo}
  \label{fig:Resultado de la prueba del servidor FTP en el firewall externo}
\end{figure}

% Nueva página
\cleardoublepage

\chapter{Configurar el firewall con unas políticas por defecto: }
Antes de empezar a configurar el firewall instalamos en el firewall interno \emph{firewalld} con el siguiente comando:
\begin{BVerbatim}
sudo yum install firewalld
\end{BVerbatim}
, y lo iniciamos con el siguiente comando: \\
\begin{BVerbatim}
sudo systemctl start firewalld
\end{BVerbatim}


Antes de configurar las políticas por defecto, hay que configurar las zonas. Esto lo haremos en el firewall interno.
\emph{firewalld} viene preconfigurado con las DMZ e interna, pero hay que agregar las redes que tenemos a esa zonas. 
Para ello, se va a utilizar los siguientes comandos:
\begin{verbatim}
firewall-cmd --zone=dmz --add-source=192.168.10.0/24
firewall-cmd --zone=internal --add-source=172.16.10.0/24
firewall-cmd --zone=external --add-interface=ens4
\end{verbatim}

En la configuración de la zona exterior ponemos la interfaz que va hacia Internet, en este caso es \emph{ens4}.
\begin{figure}[H]
  \centering
  \includegraphics[scale=0.5]{img/resultado_politicas_por_defecto.png}
  \caption{Políticas por defecto}
  \label{fig:politicas por defecto}
\end{figure}

% Nueva página
\cleardoublepage

\begin{itemize}
  \item \textbf{ACCEPT para tráfico FW a DMZ y FW a Red Interna}
  \begin{verbatim}
firewall-cmd --permanent --new-policy FWToDMZ
firewall-cmd --permanent --policy FWToDMZ --set-target ACCEPT
firewall-cmd --permanent --policy FWToDMZ --add-ingress-zone HOST
firewall-cmd --permanent --policy FWToDMZ --add-egress-zone dmz
firewall-cmd --permanent --new-policy FWToInt
firewall-cmd --permanent --policy FWToInt --set-target ACCEPT
firewall-cmd --permanent --policy FWToInt --add-ingress-zone HOST
firewall-cmd --permanent --policy FWToInt --add-egress-zone internal
  \end{verbatim}
  \begin{figure}[H]
    \centering
    \includegraphics[scale=0.55]{img/accept_fw_to_dmz_fw_Int.png}
    \caption{ACCEPT para tráfico FW a DMZ y FW a Red Interna}
    \label{fig:ACCEPT para tráfico FW a DMZ y FW a Red Interna}
  \end{figure}

  \begin{figure}[H]
    \centering
    \includegraphics[scale=0.75]{img/result_accept_fw_to_dmz_int.png}
    \caption{Resultado de ACCEPT para tráfico FW a DMZ y FW a Red Interna}
    \label{fig:Resultado de ACCEPT para tráfico FW a DMZ y FW a Red Interna}
  \end{figure}

  % Nueva pagina
  \cleardoublepage

  \item \textbf{ACCEPT para tráfico Red Interna a DMZ}
  \begin{verbatim}
firewall-cmd --permanent --new-policy IntToDMZ
firewall-cmd --permanent --policy IntToDMZ --set-target ACCEPT
firewall-cmd --permanent --policy IntToDMZ --add-ingress-zone internal
firewall-cmd --permanent --policy IntToDMZ --add-egress-zone dmz
  \end{verbatim}
  \begin{figure}[H]
    \centering
    \includegraphics[scale=0.55]{img/accept_int_to_dmz.png}
    \caption{ACCEPT para tráfico Red Interna a DMZ}
    \label{fig:ACCEPT para tráfico Red Interna a DMZ}
  \end{figure}
  \begin{figure}[H]
    \centering
    \includegraphics[scale=0.63]{img/funcionamiento_int_to_dmz.png}
    \caption{funcionamiento del ACCEPT para tráfico Red Interna a DMZ}
    \label{fig:funcionamiento ACCEPT para tráfico Red Interna a DMZ}
  \end{figure}

  % Nueva pagina
  \cleardoublepage

  \item \textbf{ACCEPT para tráfico Red Interna a Internet}
  \begin{verbatim}
firewall-cmd --permanent --new-policy IntToNet
firewall-cmd --permanent --policy IntToNet --set-target ACCEPT
firewall-cmd --permanent --policy IntToNet --add-ingress-zone internal
firewall-cmd --permanent --policy IntToNet --add-egress-zone ANY
  \end{verbatim}
  \begin{figure}[H]
    \centering
    \includegraphics[scale=0.5]{img/accept_int_to_net.png}
    \caption{ACCEPT para tráfico Red Interna a Internet}
    \label{fig:ACCEPT para tráfico Red Interna a Internet}
  \end{figure}

  \begin{figure}[H]
    \centering
    \includegraphics[scale=0.61]{img/result_accept_int_to_net.png}
    \caption{Resultado de ACCEPT para tráfico Red Interna a Internet}
    \label{fig:Resultado ACCEPT para tráfico Red Interna a Internet}
  \end{figure}

  \item \textbf{REJECT para tráfico DMZ a Red Interna e Internet a DMZ}
  \begin{verbatim}
firewall-cmd --permanent --new-policy DMZToInt
firewall-cmd --permanent --policy DMZToInt --set-target REJECT
firewall-cmd --permanent --policy DMZToInt --add-ingress-zone dmz
firewall-cmd --permanent --policy DMZToInt --add-egress-zone internal
  \end{verbatim}
  \begin{figure}[H]
    \centering
    \includegraphics[scale=0.51]{img/reject_dmz_to_int.png}
    \caption{REJECT para tráfico DMZ a Red Interna}
    \label{fig:REJECT para tráfico DMZ a Red Interna}
  \end{figure}

  \begin{figure}[H]
    \centering
    \includegraphics[scale=0.49]{img/funcionamiento_reject.png}
    \caption{Funcionamiento del REJECT para tráfico DMZ a Red Interna}
  \end{figure}

  % Nueva página
  \cleardoublepage

  \item \textbf{DROP para tráfico Internet a FW e Internet a Red Interna}
  \begin{verbatim}
firewall-cmd --permanent --new-policy NetToFW
firewall-cmd --permanent --policy NetToFW --set-target DROP
firewall-cmd --permanent --policy NetToFW --add-ingress-zone external
firewall-cmd --permanent --policy NetToFW --add-egress-zone HOST
firewall-cmd --permanent --new-policy NetToInt
firewall-cmd --permanent --policy NetToInt --set-target DROP
firewall-cmd --permanent --policy NetToInt --add-ingress-zone external
firewall-cmd --permanent --policy NetToInt --add-egress-zone internal
  \end{verbatim}

  \begin{figure}[H]
    \centering
    \includegraphics[scale=0.58]{img/drop_net_to_network.png}
    \caption{DROP para tráfico Internet a Red Interna e Internet a FW}
  \end{figure}

\end{itemize}

% Nueva página
\cleardoublepage

\chapter{Configurar reglas utilizando Macros para permitir el tráfico necesario}
\begin{itemize}
  \item \textbf{Tráfico DNS para la resolución de nombres al servidor DNS externo} 
  \\ \\
  \begin{BVerbatim}
firewall-cmd --permanent --policy NetToInt --add-service=dns
firewall-cmd --permanent --policy NetToFW --add-service=dns
  \end{BVerbatim} 
  \\ \\
  Luego en el firewall externo se añade la siguiente regla al fichero \emph{/etc/shorewall/rules}:
  \begin{verbatim}
#ACTION         SOURCE          DEST                    PROTO           DPORT
ACCEPT          net             dmz                     udp             53
ACCEPT          net             loc                     udp             53
  \end{verbatim}

  \begin{figure}[H]
    \centering
    \includegraphics[scale=0.75]{img/dns.png}
    \caption{Tráfico DNS para la resolución de nombres al servidor DNS externo}
  \end{figure}

  \begin{figure}[H]
    \centering
    \includegraphics[scale=0.65]{img/dns_ull.png}
    \caption{Funcionamiento del tráfico DNS para la resolución de nombres al servidor DNS externo}
  \end{figure}

  % Nueva página
  \cleardoublepage

  \item \textbf{Tráfico de cualquier tipo desde la red interna a servidores de Internet}
  \begin{figure}[H]
    \centering
    \includegraphics[scale=0.75]{img/internet_on_client.png}
    \caption{\emph{PING} a google.es desde el cliente}
  \end{figure}

  \begin{figure}[H]
    \centering
    \includegraphics[scale=0.6]{img/links_google.png}
    \caption{Acceso a google.es desde el cliente mediante el navegador \emph{links}}
  \end{figure}

  % Nueva página
  \cleardoublepage

  \item \textbf{Tráfico Web y FTP desde Internet al servidor web}
  
  En el firewall externo, en el fichero \emph{/etc/shorewall/rules} se añade la siguiente regla:
  \begin{verbatim}
#ACTION         SOURCE          DEST                    PROTO           DPORT
DNAT            net             dmz:192.168.10.100      tcp             20
DNAT            net             dmz:192.168.10.100      tcp             21
DNAT            net             dmz:192.168.10.100      tcp             80
ACCEPT          net             dmz:192.168.10.100      tcp             http,ftp
  \end{verbatim}

  \begin{figure}[H]
    \centering
    \includegraphics[scale=0.65]{img/nginx_navegador.png}
    \caption{Acceso al servidor web de la DMZ desde un navegador}
  \end{figure}

  \item \textbf{Tráfico Web desde la red Interna al servidor web de la DMZ}
  \begin{verbatim}
firewall-cmd --permanent --policy IntToDMZ --add-service=http
  \end{verbatim}
  Y en el firewall externo, en el fichero \emph{/etc/shorewall/rules} se añade la siguiente regla:
  \\ \\
  \begin{BVerbatim}[fontsize=\fontsize{9}{10}]
#ACTION         SOURCE          DEST                    PROTO           DPORT   SPORT   ORIGDEST
DNAT            loc             dmz:192.168.10.100      tcp             80      -       &NET_IF
  \end{BVerbatim}

  \begin{figure}[H]
    \centering
    \includegraphics[scale=0.655]{img/http.png}
    \caption{Tráfico Web desde la red Interna al servidor web de la DMZ}
  \end{figure}

  \begin{figure}[H]
    \centering
    \includegraphics[scale=0.655]{img/nginx_cliente.png}
    \caption{Acceso al servidor web de la DMZ desde el cliente con direccion publica y privada}
  \end{figure}

  % Nueva página
  \cleardoublepage

  \item \textbf{Configuracion final del \emph{Shorewall} en los ficheros \emph{rules} y \emph{policy}}
  \begin{itemize}
    \item \textbf{Fichero \emph{rules}} \\ \\
    \begin{BVerbatim}[fontsize=\fontsize{9}{10}]
#ACTION         SOURCE          DEST                    PROTO           DPORT   SPORT   ORIGDEST
ACCEPT          net             fw                      tcp             ssh
ACCEPT          fw              dmz                     tcp             ssh
ACCEPT          fw              loc                     tcp             ssh
DNAT            net             dmz:192.168.10.100      tcp             20
DNAT            net             dmz:192.168.10.100      tcp             21
DNAT            net             fw:192.168.10.2         tcp             22
DNAT            net             dmz:192.168.10.100      tcp             80
DNAT            loc             dmz:192.168.10.100      tcp             80      -       &NET_IF
ACCEPT          net             dmz                     udp             53
ACCEPT          net             loc                     udp             53
ACCEPT          loc             net
ACCEPT          net             dmz:192.168.10.100      tcp             http,ftp
ACCEPT          loc             dmz:192.168.10.100      tcp             http
    \end{BVerbatim}

    \item \textbf{Fichero \emph{policy}}
    \begin{verbatim}
#SOURCE         DEST            POLICY          LOG LEVEL
fw              dmz             ACCEPT          info
fw              loc             ACCEPT          info
fw              net             ACCEPT
loc             net             ACCEPT          info
loc             fw              ACCEPT          info
loc             dmz             ACCEPT          info
net             dmz             REJECT          info
net             fw              DROP            info
net             loc             REJECT          info
dmz             fw              ACCEPT          info
dmz             loc             ACCEPT          info
dmz             net             ACCEPT          info
    \end{verbatim}
  \end{itemize}

\end{itemize}

% Nueva página
\cleardoublepage

\chapter{Bibliografía} % En formato APA
\begin{enumerate}
\item Oliveros, D. (2013, 14 de marzo). Configurar Shorewall en Debian. Dayron Oliveros. Recuperado de \url{https://www.youtube.com/watch?v=20E0QxWwAlk}
\item Thomas M. Eastep. (2020). snat — Shorewall SNAT/Masquerade definition file. Shorewall. Recuperado de \url{https://shorewall.org/manpages/shorewall-snat.html}
\item Thomas M. Eastep. (2020). interfaces — Shorewall interfaces file. Shorewall. Recuperado de \url{https://shorewall.org/manpages/shorewall-interfaces.html}
\item Luz, S. (2023). Servidor FTP ProFTPd para Linux: Instalación y configuración. Redes Zone. Recuperado de \url{https://www.redeszone.net/tutoriales/servidores/proftpd/}
\item Alonsojpd. (2022). Solución al error Failed to download metadata for repo appstream en CentOS 8. Proyectoa. Recuperado de \url{https://proyectoa.com/solucion-al-error-failed-to-download-metadata-for-repo-appstream-en-centos-8/}
\item firewalld. (s.f.). Concepts and Configuration. firewalld. Recuperado de \url{https://firewalld.org/documentation/concepts.html}
\end{enumerate}

\end{document}